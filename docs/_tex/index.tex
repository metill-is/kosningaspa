% Options for packages loaded elsewhere
\PassOptionsToPackage{unicode}{hyperref}
\PassOptionsToPackage{hyphens}{url}
\PassOptionsToPackage{dvipsnames,svgnames,x11names}{xcolor}
%
\documentclass[
  letterpaper,
  DIV=11,
  numbers=noendperiod]{scrartcl}

\usepackage{amsmath,amssymb}
\usepackage{iftex}
\ifPDFTeX
  \usepackage[T1]{fontenc}
  \usepackage[utf8]{inputenc}
  \usepackage{textcomp} % provide euro and other symbols
\else % if luatex or xetex
  \usepackage{unicode-math}
  \defaultfontfeatures{Scale=MatchLowercase}
  \defaultfontfeatures[\rmfamily]{Ligatures=TeX,Scale=1}
\fi
\usepackage{lmodern}
\ifPDFTeX\else  
    % xetex/luatex font selection
\fi
% Use upquote if available, for straight quotes in verbatim environments
\IfFileExists{upquote.sty}{\usepackage{upquote}}{}
\IfFileExists{microtype.sty}{% use microtype if available
  \usepackage[]{microtype}
  \UseMicrotypeSet[protrusion]{basicmath} % disable protrusion for tt fonts
}{}
\makeatletter
\@ifundefined{KOMAClassName}{% if non-KOMA class
  \IfFileExists{parskip.sty}{%
    \usepackage{parskip}
  }{% else
    \setlength{\parindent}{0pt}
    \setlength{\parskip}{6pt plus 2pt minus 1pt}}
}{% if KOMA class
  \KOMAoptions{parskip=half}}
\makeatother
\usepackage{xcolor}
\setlength{\emergencystretch}{3em} % prevent overfull lines
\setcounter{secnumdepth}{-\maxdimen} % remove section numbering
% Make \paragraph and \subparagraph free-standing
\makeatletter
\ifx\paragraph\undefined\else
  \let\oldparagraph\paragraph
  \renewcommand{\paragraph}{
    \@ifstar
      \xxxParagraphStar
      \xxxParagraphNoStar
  }
  \newcommand{\xxxParagraphStar}[1]{\oldparagraph*{#1}\mbox{}}
  \newcommand{\xxxParagraphNoStar}[1]{\oldparagraph{#1}\mbox{}}
\fi
\ifx\subparagraph\undefined\else
  \let\oldsubparagraph\subparagraph
  \renewcommand{\subparagraph}{
    \@ifstar
      \xxxSubParagraphStar
      \xxxSubParagraphNoStar
  }
  \newcommand{\xxxSubParagraphStar}[1]{\oldsubparagraph*{#1}\mbox{}}
  \newcommand{\xxxSubParagraphNoStar}[1]{\oldsubparagraph{#1}\mbox{}}
\fi
\makeatother


\providecommand{\tightlist}{%
  \setlength{\itemsep}{0pt}\setlength{\parskip}{0pt}}\usepackage{longtable,booktabs,array}
\usepackage{calc} % for calculating minipage widths
% Correct order of tables after \paragraph or \subparagraph
\usepackage{etoolbox}
\makeatletter
\patchcmd\longtable{\par}{\if@noskipsec\mbox{}\fi\par}{}{}
\makeatother
% Allow footnotes in longtable head/foot
\IfFileExists{footnotehyper.sty}{\usepackage{footnotehyper}}{\usepackage{footnote}}
\makesavenoteenv{longtable}
\usepackage{graphicx}
\makeatletter
\def\maxwidth{\ifdim\Gin@nat@width>\linewidth\linewidth\else\Gin@nat@width\fi}
\def\maxheight{\ifdim\Gin@nat@height>\textheight\textheight\else\Gin@nat@height\fi}
\makeatother
% Scale images if necessary, so that they will not overflow the page
% margins by default, and it is still possible to overwrite the defaults
% using explicit options in \includegraphics[width, height, ...]{}
\setkeys{Gin}{width=\maxwidth,height=\maxheight,keepaspectratio}
% Set default figure placement to htbp
\makeatletter
\def\fps@figure{htbp}
\makeatother

\KOMAoption{captions}{tableheading}
\makeatletter
\@ifpackageloaded{caption}{}{\usepackage{caption}}
\AtBeginDocument{%
\ifdefined\contentsname
  \renewcommand*\contentsname{Table of contents}
\else
  \newcommand\contentsname{Table of contents}
\fi
\ifdefined\listfigurename
  \renewcommand*\listfigurename{List of Figures}
\else
  \newcommand\listfigurename{List of Figures}
\fi
\ifdefined\listtablename
  \renewcommand*\listtablename{List of Tables}
\else
  \newcommand\listtablename{List of Tables}
\fi
\ifdefined\figurename
  \renewcommand*\figurename{Figure}
\else
  \newcommand\figurename{Figure}
\fi
\ifdefined\tablename
  \renewcommand*\tablename{Table}
\else
  \newcommand\tablename{Table}
\fi
}
\@ifpackageloaded{float}{}{\usepackage{float}}
\floatstyle{ruled}
\@ifundefined{c@chapter}{\newfloat{codelisting}{h}{lop}}{\newfloat{codelisting}{h}{lop}[chapter]}
\floatname{codelisting}{Listing}
\newcommand*\listoflistings{\listof{codelisting}{List of Listings}}
\makeatother
\makeatletter
\makeatother
\makeatletter
\@ifpackageloaded{caption}{}{\usepackage{caption}}
\@ifpackageloaded{subcaption}{}{\usepackage{subcaption}}
\makeatother

\ifLuaTeX
  \usepackage{selnolig}  % disable illegal ligatures
\fi
\usepackage{bookmark}

\IfFileExists{xurl.sty}{\usepackage{xurl}}{} % add URL line breaks if available
\urlstyle{same} % disable monospaced font for URLs
\hypersetup{
  colorlinks=true,
  linkcolor={blue},
  filecolor={Maroon},
  citecolor={Blue},
  urlcolor={Blue},
  pdfcreator={LaTeX via pandoc}}


\author{}
\date{}

\begin{document}


\section{Introduction}\label{introduction}

There is considerable public interest in election forecasting, a task
that is complicated by the numerous mechanisms that may lead to a
divergence between voting intentions as measured by pre-election polls
and election results. Polling error refers to the accuracy of
pre-election polls in terms of predicting election results but this is
bound to be affected lead, referring to the time between the
administration of the poll and the election {[}@jennings2020election{]}.
Simply put, we should expect that the fewer days that remain until the
election is held, the more accurate the polls and election forecasts
will become, although even in the final days of the campaign
considerable error, far exceeding sampling error alone, can be expected
{[}@jennings2016timeline; @jennings2018election{]}.
@jennings2020election suggest that a lead of two or three months before
an election occurs is often sufficient for creating accurate election
forecasts.

However, the accuracy and lead of election forecasts may vary from
country to country, as specific polling error mechanisms may be at work
in some contexts but not in others. Survey methodologists rely on the
Total Survey Error framework to decompose error sources which cause
survey statistics to diverge from population parameters. Of these error
sources, sampling error is almost always accounted for, but non-sampling
errors such as coverage, nonresponse, measurement, and estimation errors
often are not, although each can cause both systematic and variable
errors {[}@biemer2010total; @groves2010total{]}. @prosser2018twilight
find that measurement error (``shy'' voters) plays a limited role in
polling error. Instead, the fact that voting intention is dynamic and
problems with gathering representative samples and weighting are the
primary causes of polling error. This suggest that nonresponse and
estimation errors play important roles in polling error.

Nonresponse bias is the prime suspect in analyses of polling error and
can be caused by various mechanisms. On the one hand, If the same
characteristic predicts both response propensities and target variables
(i.e.~voting intentions), corresponding to a missing at random
mechanism, bias can be adjusted in estimation by weighting on the
characteristic, provided that it has been observed. On the other hand,
if a target variable is the true cause of variation in response
propensities, corresponding to a missing not at random mechanism,
analyses of the target variable will be biased by nonresponse bias
{[}@little2020statistical; @groves2006nonresponse{]}. The latter
scenario is easy to envisage in polls, where voters of specific parties
may be more engaged in politics even when weighting adjustments for
other characteristics are applied.

To address the issue of polling error, election forecasters often adjust
for ``house effects'', where some pollsters systematically over- or
underestimate support for specific parties, relative to other pollsters
{[}@jennings2018election{]}. Here, we add to the literature on election
forecasting by adjusting for industry effects, referring to cases where
polling error has been observed repeatedly and the direction in which it
occurs as it relates to specific parties is consistent. This scenario
would reflect an industry wide inability to recruit representative
samples or to identify suitable auxiliary information to adjust for
nonresponse bias as it relates to voting intention
{[}@kalton2003weighting; @groves2006nonresponse{]}.

We put our assumptions to an empirical test by publishing an election
forecast prior to the Icelandic parliamentary election of 2024. Response
rates in Icelandic surveys have declined significantly over time but
remain high in the international contexts \textbf{(Hvaða Einarsson er
hvað í bibtex?)} (Einarsson et al., n.d.; Einarsson and Helgason, n.d.).
However, most Icelandic pre-election polls are conducted using
probability-based online panels. Online panels have mixed records when
it comes to polling error but probability-based ones have been found to
outperform non-probability ones {[}@kennedy2016evaluation;
@callegaro2014online{]}. However, the repeated selection of respondents
carries significant risks in terms of panel conditioning
{[}@struminskaya2021panel{]} and attrition {[}@frankel2014looking{]}.
Therefore, online panels often rely heavily on model-based (rather than
design-based) inference {[}@little2004model{]}, which will only be
successful in the case that a suitable source of auxiliary information
is identified.

Icelandic politics were characterised by remarkable stability prior to
the 2008 financial crisis but have seen high electoral volatility since
{[}@onnudottir2021electoral; @helgason2022electoral{]} . In fact, only
one of the previous four governments has been able to serve a full
four-year term, causing early elections in each instance, for a total of
five elections in the span of 11 years. Each election has been
associated with polling errors which can be attributed to the high
degree of electoral volatility. Despite this, there has been little
variation in voting intention estimates between polling houses and
biases at the party level have been similar election-to-election, with
right-wing parties being underestimated \textbf{(Hvaða bibtex er
þetta?)} (Einarsson and Helgason, n.d.). This suggests that pollsters
are relying on similar methods but failing to address the issue of
polling error, i.e.~industry effects may be at play.

As our model comes with significant assumptions regarding the direction
of polling error, it is plausible that it may not improve the prediction
of election outcomes. For example, if pollsters have identified new
recruitment methods or identified weighting characteristics correlated
both with the propensity to respond and target variables
{[}@little2005weighting{]}, our adjustments will introduce bias rather
than adjust it. If, however, the same industry wide problems remain, our
model will provide a more accurate picture of the voting intentions of
the Icelandic electorate than an unadjusted polling average would.

\section{Methods}\label{methods}

We develop a Bayesian hierarchical model to forecast the 2024 Icelandic
Parliamentary Elections by combining:

\begin{itemize}
\tightlist
\item
  A dynamic linear model for polling data that captures:

  \begin{itemize}
  \tightlist
  \item
    Temporal evolution of party support
  \item
    House-specific and industry-wide polling biases
  \item
    Increased volatility during the government coalition change leading
    up to the election
  \item
    Cross-party correlations in support changes
  \end{itemize}
\item
  A fundamentals model incorporating:

  \begin{itemize}
  \tightlist
  \item
    Historical election results
  \item
    Incumbency status and duration
  \item
    Economic indicators (inflation and GDP growth)
  \end{itemize}
\end{itemize}

\subsection{Model Structure}\label{model-structure}

\subsubsection{Statistical Framework}\label{statistical-framework}

The model combines polling data \(y_{n,p}\) and fundamentals data
through a hierarchical structure:

\[
\begin{aligned}
y_{n} &\sim \text{Dirichlet-Multinomial}(\pi_{n}, \phi_{h[n]}) \\
\pi_{n} &= \text{softmax}(\eta_{n}) \\
\eta_{n,p} &= \beta_{p,t[n]} + \gamma_{p,h[n]}
\end{aligned}
\]

where \(\phi_{h[n]}\) is house-specific overdispersion, \(t[n]\) indexes
the time of poll \(n\), and \(h[n]\) indexes the polling house.

The fundamentals component predicts election results through:

\[
\begin{aligned}
y^{(f)}_{d} &\sim \text{Dirichlet-Multinomial}(\pi^{(f)}_{d}, \phi_f) \\
\pi^{(f)}_{d} &= \text{softmax}(\mu_{d}) \\
\mu_{d,p} &= \alpha_p + \beta_{\text{lag}}x_{p,d} + \beta_{\text{years}}\log(I_{p,d}) + \beta_{\text{vnv}}v_{p,d} + \beta_{\text{growth}}g_{p,d}
\end{aligned}
\]

The two components are linked through the election day prediction
\(\mu_{\text{pred}}\), which is calculated from the fundamentals model
using current economic conditions and previous election results. This
prediction serves as a prior for the election day support:

\[
\beta_{T} \sim \mathcal{N}(\mu_{\text{pred}}, \tau_f\cdot \sigma)
\]

The relative weight between polling and fundamentals is controlled by
\(\tau_f\), which is computed so that the fundamentals component has a
desired percentage weight in the prediction at a specified time before
the election (see Appendix for details).

\subsubsection{Polling Component}\label{polling-component}

The polling model tracks latent party support \(\beta_{p,t}\) through
time using a multivariate random walk with correlation structure
\(\Omega\) between parties:

\[
\beta_{t} = \begin{cases}
\mathrm{Normal}\left(\beta_{t+1}, (1 + \tau_s)\sqrt{\Delta_t} \boldsymbol \Sigma\right) & \text{after government split} \\
\mathrm{Normal}\left(\beta_{t+1}, \sqrt{\Delta_t} \boldsymbol \Sigma\right) & \text{otherwise}
\end{cases}
\]

where:

\begin{itemize}
\tightlist
\item
  \(\boldsymbol \Sigma = \text{diag}(\sigma_1,\ldots,\sigma_P) \Omega \text{diag}(\sigma_1,\ldots,\sigma_P)\)
  is the covariance matrix
\item
  \(\sigma_p\) captures party-specific volatility scales
\item
  \(\Omega\) is the correlation matrix between the party-specific
  innovations
\item
  \(\tau_s\) allows for increased volatility after the government split
\item
  \(\Delta_t\) is the number of days between polls
\end{itemize}

The observed polling data follows a Dirichlet-multinomial distribution
with house-specific overdispersion parameters \(\phi_h\):

\[
\begin{aligned}
y_n &\sim \text{Dirichlet-Multinomial}(\pi_n, \phi_{h[n]}) \\
\pi_n &= \text{softmax}(\beta_{t[n]} + \gamma_{h[n]})
\end{aligned}
\]

where \(\gamma_{h}\) captures house-specific biases in reported support
levels, with the election day house effect fixed to zero
(\(\gamma_1 = 0\)). The other house effects are given a hierarchical
prior:

\[
\begin{aligned}
\gamma_{p,h} &= \mathrm{Normal}\left(\mu_{\gamma,p}, \sigma_{\gamma,p} \right)\\
\mu_{\gamma,p} &\sim \mathcal{N}(0, 1) \\
\sigma_{\gamma,p} &\sim \text{Exponential}(1) \\
\sum_{p=2}^{P} \mu_{\gamma,p} &\sim \mathcal{N}(0, 1) \\
\sum_{p=2}^{P} \gamma_{p, h} &\sim \mathcal{N}(0, 1)
\end{aligned}
\]

\subsubsection{Fundamentals Component}\label{fundamentals-component}

The fundamentals model predicts party vote shares using economic and
political variables:

\[
\pi = \text{softmax}(\alpha + \beta_\text{lag}x + \beta_\text{inc}\log(I) + \beta_\text{infl}V + \beta_\text{growth}G)
\]

where:

\begin{itemize}
\tightlist
\item
  \(\alpha_p\) are party-specific intercepts that sum to zero
  (\(\sum_p \alpha_p = 0\))
\item
  \(x\) are previous election vote shares
\item
  \(I\) are years in government for incumbent parties
\item
  \(V\) is inflation on an annual basis six months before the election
\item
  \(G\) is economic growth on an annual basis six months before the
  election
\item
  \(\beta_\text{lag}\) captures persistence in party support
\item
  \(\beta_\text{inc}\) measures the effect of time spent in government
  on incumbent parties
\item
  \(\beta_\text{vnv}\) captures the impact of inflation on incumbent
  parties
\item
  \(\beta_\text{growth}\) captures the impact of growth on incumbent
  parties
\end{itemize}

\subsubsection{Model Integration}\label{model-integration}

The fundamentals prediction serves as a prior for the election day vote
shares \(\beta_T\):

\[
\beta_T \sim \mathcal{N}(\mu_\text{pred}, \tau_f \cdot \sigma)
\]

where \(\mu_\text{pred}\) is the fundamentals prediction and \(\tau_f\)
controls how much weight is given to the fundamentals versus polling
data at some point before the elections (see Appendix for details on
choosing \(\tau_f\)).

\subsubsection{Constituency Component}\label{constituency-component}

The model extends the national-level polling model to incorporate
constituency-level effects through a hierarchical structure. This allows
us to capture systematic differences in party support across
constituencies while sharing information between them.

\paragraph{Constituency Effects}\label{constituency-effects}

The constituency-level model adds constituency-specific deviations
\(\delta_{p,k}\) to the national-level support:

\[
\begin{aligned}
y_{n,k} &\sim \text{Dirichlet-Multinomial}(\pi_{n,k}, \phi_{h[n]}) \\
\pi_{n,k} &= \text{softmax}(\eta_{n,k}) \\
\eta_{n,k,p} &= \beta_{p,t[n]} + \gamma_{p,h[n]} + \delta_{p,k}
\end{aligned}
\]

where \(k\) indexes constituencies. The constituency effects follow a
hierarchical prior:

\[
\begin{aligned}
\delta_{p,k} &\sim \text{Normal}(0, \sigma_{\delta,p}) \\
\sigma_{\delta,p} &\sim \text{Exponential}(1)
\end{aligned}
\]

This structure allows for:

\begin{itemize}
\tightlist
\item
  Constituency-specific deviations from national support levels
\item
  Party-specific scales of constituency variation
  (\(\sigma_{\delta,p}\))
\item
  Partial pooling of information across constituencies
\item
  Separate treatment of each party's geographic distribution
\end{itemize}

The constituency effects are constrained to sum to zero across
constituencies for each party to ensure identifiability:

\[
\sum_{k=1}^K \delta_{p,k} = 0
\]

This constraint also prevents the constituency effects from absorbing
national-level trends and ensures that the deviations represent relative
differences across constituencies.

\paragraph{Integration with National
Model}\label{integration-with-national-model}

The constituency model is integrated with the national model through:

\begin{enumerate}
\def\labelenumi{\arabic{enumi}.}
\tightlist
\item
  Shared national-level parameters (\(\beta_{p,t}\))
\item
  Common house effects (\(\gamma_{p,h}\))
\item
  Joint estimation of overdispersion parameters (\(\phi_h\))
\end{enumerate}

This allows constituency-level polls to inform national-level estimates
and vice versa, while accounting for systematic differences between
constituencies.

\paragraph{Prediction}\label{prediction}

The model generates both national-level predictions and
constituency-specific predictions:

\begin{itemize}
\tightlist
\item
  National: \(y^{\text{rep}}_{\text{national}}\) represents overall
  party support
\item
  Constituency: \(y^{\text{rep}}_{\text{constituency}}\) gives
  constituency-level predictions
\end{itemize}

These predictions incorporate the combined uncertainty in:

\begin{itemize}
\tightlist
\item
  National trends
\item
  Constituency-specific effects
\item
  House effects
\item
  Overdispersion in both national and constituency-level polls
\end{itemize}

\section{Results}\label{results}

\subsection{Parameter Estimates}\label{parameter-estimates}

\subsubsection{Polling Component}\label{polling-component-1}

\paragraph{House Effects}\label{house-effects}

\begin{itemize}
\tightlist
\item
  Plot of house-specific biases (\(\gamma_{h}\)) with uncertainty
  intervals
\item
  Table of significant house effects by party
\item
  Analysis of industry-wide bias patterns
\end{itemize}

\paragraph{Volatility Parameters}\label{volatility-parameters}

\begin{itemize}
\tightlist
\item
  Estimates of party-specific volatility (\(\sigma_p\))
\item
  Impact of government split (\(\tau_\text{stjornarslit}\))
\item
  Systematic shifts after split (\(\beta_\text{stjornarslit}\))
\end{itemize}

\paragraph{Cross-Party Correlations}\label{cross-party-correlations}

\begin{itemize}
\tightlist
\item
  Heatmap of correlation matrix (\(\Omega\))
\item
  Discussion of strongest party-to-party relationships
\item
  Clustering analysis of correlated party groups
\end{itemize}

\subsubsection{Fundamentals Component}\label{fundamentals-component-1}

\paragraph{Economic Effects}\label{economic-effects}

\begin{itemize}
\tightlist
\item
  Coefficient plot for \(\beta_\text{growth}\)
\item
  Analysis of GDP growth impact on incumbent parties
\end{itemize}

\paragraph{Political Effects}\label{political-effects}

\begin{itemize}
\tightlist
\item
  Coefficient estimates for:

  \begin{itemize}
  \tightlist
  \item
    Previous vote share effect (\(\beta_\text{lag}\))
  \item
    Incumbency duration (\(\beta_\text{inc}\))
  \item
    Government participation (\(\beta_\text{vnv}\))
  \end{itemize}
\item
  Discussion of which factors most strongly predict party support
\end{itemize}

\subsection{Model Validation}\label{model-validation}

\subsubsection{Historical Performance}\label{historical-performance}

\begin{itemize}
\tightlist
\item
  Out-of-sample predictions for previous elections
\item
  Comparison with simple polling averages
\item
  Analysis of when/where model performs best/worst
\end{itemize}

\subsubsection{Uncertainty Calibration}\label{uncertainty-calibration}

\begin{itemize}
\tightlist
\item
  Coverage of prediction intervals
\item
  Comparison of predicted vs.~actual volatility
\item
  Assessment of overdispersion estimates (\(\phi_h\))
\end{itemize}

\subsection{2024 Election Prediction}\label{election-prediction}

\subsubsection{Point Predictions}\label{point-predictions}

\begin{itemize}
\tightlist
\item
  Table of predicted vote shares with 95\% intervals
\item
  Comparison to latest polls
\item
  Discussion of largest predicted changes
\end{itemize}

\subsubsection{Coalition Scenarios}\label{coalition-scenarios}

\begin{itemize}
\tightlist
\item
  Probability of different majority combinations
\item
  Most likely government formations
\item
  Key parties for coalition formation
\end{itemize}

\subsubsection{Prediction Evolution}\label{prediction-evolution}

\begin{itemize}
\tightlist
\item
  Plot showing how predictions changed over time
\item
  Impact of recent polling data
\item
  Effect of economic/political developments
\end{itemize}

\section{Discussion}\label{discussion}

\section{Appendix}\label{appendix}

\subsection{Model Specification
Details}\label{model-specification-details}

\subsubsection{Notation}\label{notation}

\paragraph{Input Data}\label{input-data}

\subparagraph{Polling Data}\label{polling-data}

\begin{itemize}
\tightlist
\item
  \(P\): Number of political parties \emph{(including the Other
  category)}
\item
  \(T\): Number of time points (dates), \(1, \dots, T\), where \(T\) is
  the date of the next election
\item
  \(H\): Number of polling houses
\item
  \(N\): Number of observations (polls)
\item
  \(y_{n,p}\): Count of responses for party \(p\) in poll \(n\)
\item
  \(\Delta_t\): The time difference between polls at \(t-1\) and \(t\)
  in days
\end{itemize}

\subparagraph{Fundamentals Data}\label{fundamentals-data}

\begin{itemize}
\tightlist
\item
  \(D_f\): Number of past elections
\item
  \(P_f\): Number of parties in historical data
\item
  \(y_{p,d}^{(f)}\): Vote share for party \(p\) in election \(d\)
\item
  \(x_{p,d}\): Previous vote share for party \(p\) in election \(d\)
\item
  \(I_{p,d}\): Years party \(p\) has been incumbent at election \(d\)
\item
  \(v_{p,d}\): Inflation rate (\%) for party \(p\) at election \(d\) (if
  incumbent)
\item
  \(g_{p,d}\): GDP growth rate (\%) for party \(p\) at election \(d\)
  (if incumbent)
\end{itemize}

\subsubsection{Data Preprocessing}\label{data-preprocessing}

The economic variables are transformed to better capture their effects:

\begin{itemize}
\tightlist
\item
  Incumbent years are log-transformed: \(\log(I_{p,d})\)
\item
  Excess inflation is calculated as deviation from 2\% target and
  log-transformed: \(\log(1 + v_{p,d} - 0.02)\)
\item
  GDP growth is log-transformed: \(\log(1 + g_{p,d}/100)\)
\end{itemize}

Both economic variables (inflation and growth) are only included for
incumbent parties by multiplying them with the incumbency indicator.

\subsubsection{Government Split Effects}\label{government-split-effects}

The model accounts for potential changes in voting patterns during
government coalition changes through two mechanisms: 1. Increased
volatility scaled by \(\tau_\text{stjornarslit}\) 2. Potential
systematic shifts in party support through \(\beta_\text{stjornarslit}\)

\subsubsection{Cross-Party
Correlations}\label{cross-party-correlations-1}

Rather than treating each party's support as independent, the model
captures correlations in support changes through a Cholesky
decomposition of the correlation matrix \(\Omega\). This allows the
model to account for situations where increased support for one party
typically corresponds to decreased support for specific other parties.

\subsubsection{\texorpdfstring{Choosing
\(\tau_f\)}{Choosing \textbackslash tau\_f}}\label{choosing-tau_f}

To choose how we want to calculate the standard deviation for prior on
\(\beta_T\), we can frame our model as a Gaussian-Gaussian conjugate
problem where:

\begin{itemize}
\tightlist
\item
  The prior (fundamentals prediction) is:
  \(\beta_T \sim \mathcal{N}(\mu_{\text{pred}}, \tau_f \cdot \sigma)\)
\item
  The likelihood (polling prediction from time t) is:
  \(\beta_T \sim \mathcal{N}(\beta_t, V(t) \cdot \sigma)\)
\end{itemize}

where \(V(t)\) represents the accumulated variance from time t to
election time T:

\begin{itemize}
\tightlist
\item
  For \(t \leq 47\) (after government split):
  \(V(t) = t \cdot (1 + \tau_{\text{stjornarslit}})^2\)
\item
  For t \textgreater{} 47 (before government split):
  \(V(t) = (t - 47) + 47 \cdot (1 + \tau_{\text{stjornarslit}})^2\)
\end{itemize}

Using standard Gaussian-Gaussian conjugate formulas:

\[
\begin{aligned}
\text{Posterior precision} &= \frac{1}{(\tau_f \cdot \sigma)^2} + \frac{1}{V(t) \cdot \sigma^2} \\
&= \left(\frac{1}{\tau_f^2} + \frac{1}{V(t)}\right) \cdot \frac{1}{\sigma^2}
\end{aligned}
\]

For a desired fundamentals weight w at time t:

\[
w = \frac{1/\tau_f^2}{1/\tau_f^2 + 1/V(t)}
\]

Solving for \(\tau_f\):

\[
\tau_f = \sqrt{V(t) \cdot (1-w)/w}
\]

where \(V(t)\) depends on \(\tau_{\text{stjornarslit}}\) as defined
above. As an example, if we choose \(w = \frac13\) and \(t = 180\), we
get:

\[
\begin{aligned}
V(180) &= (180 - 47) + 47 \cdot (1 + \tau_{\text{stjornarslit}})^2 \\
&= 133 + 47 \cdot (1 + \tau_{\text{stjornarslit}})^2
\end{aligned}
\]

Then:

\[
\begin{aligned}
\tau_f &= \sqrt{V(180) \cdot (1-\frac{1}{3})/\frac{1}{3}} \\
&= \sqrt{(133 + 47 \cdot (1 + \tau_{\text{stjornarslit}})^2) \cdot 2} \\
&= \sqrt{266 + 94 \cdot (1 + \tau_{\text{stjornarslit}})^2}
\end{aligned}
\]

This shows how \(\tau_f\) adapts to the estimated value of
\(\tau_{\text{stjornarslit}}\) in our model, increasing when there is
more uncertainty during the government split period.




\end{document}
